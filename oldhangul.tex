% !TEX program = xelatex
\documentclass{oblivoir}

\usepackage{pmhanguljamo}
\setmainhangulfont{Noto Serif CJK KR}[Script=Hangul,Renderer=OpenType]

\title{옛한글 풀이}
\author{이재호}

\begin{document}
\maketitle

\begin{enumerate}[\bfseries 1.]
  \item 다음 단어를 조판하시오.
    \begin{enumerate}[(i)]
      \item \jamoword{h@n/gvr}
      \item \jamoword{na/ras;/mar:/ss@/mi;}
      \item \jamoword{m@iq/j@/wi g@r/x@/sya/d@i}
      \item \jamoword{h@n/je ob/se/yey}
    \end{enumerate}

  \item \href{http://wiki.ktug.org/wiki/wiki.php/훈민정음언해}{\sffamily KTUG Wiki: 훈민정음언해} 페이지에서
    \texttt{hunminjeongeum.tex}을 컴파일하여 보시오.\par
    완료.

  \item 다음에 주어진 텍스트를
    \begin{enumerate}[(1)]
      \item pmhanguljamo 형식의 입력으로 변환하시오.\par\noindent
      \item 한글과 컴퓨터에서 제공하는 ``훈민정음 세로쓰기체''를 다운로드하여 이 폰트로 변환한 텍스트를 식자하시오.
        \url{https://font.hancom.com/sub1_3.html}
    \end{enumerate}
    \framebox{\adhochangulfont{Hancom Hoonminjeongeum_V}%
      \parbox{.94\textwidth}{%
        \jamoword{%
          ir/ir/vn kv/gyey sg@i/cye yey/sa/ram/vr s@ix/gag/h@/go ga/gun/vr
          jyex/ib/h@/ya yes/ja/o/d@i gox/syun/i h@/nan/ma/ri ye/bo/si/yo
          jyen/s@ix/vi mu/sam vn/hyey sgis/cye/den/ji i/s@ix/vi bu/bu/doy/ya
          cax/gi/h@ix/sir da ba/ri/go yey/mo/do syux/syax/h@/go ye/gox/do
          sim/svs/ges/man mu/sam/joy/ga jin/jyux/h@/ya ir/jyem/hyey/ryug
          eb/sye/sv/ni/, yug/cin/mu/jog u/ri sin/syey syen/yex/h@ix/hoa
          nue/ra h@/mye sa/hu/gam/jax e/i/ha/ri myex/san/d@i/car/vy sin/go/i/na
          h@/ya nam/ye/gan nas/ke/dv/myen pyex/s@ix han/vr pur/ge/si/ni
          ga/gun/vi sdvs/si es/de/h@/o}}}

  \item 김소월의 진달래꽃을 1925년 초판 출간 당시의 표기법과 띄어쓰기로 조판하시오.
    가로쓰기 합니다.\par
    \jamoword{%
      na/bo/gi/ga yeg/gye/ue/\\
      ga/sir/sday/ey/nvn/\\
      mar/eb/si go/hi bo/nay/dv/ri/u/ri/da/\\
      \\
      寧邊ey/藥山\\
      jin/dar/nay/sgos/\\
      a/rvm/sda/da ga/sir/gir/ey sbu/ri/u/ri/da/\\
      \\
      ga/si/nvn/ge/rvm/ge/rvm/\\
      no/hin/gv/sgos/cvr/\\
      sab/bun/hi/jv/rye/barb/go ga/si/ob/so/se/\\
      \\
      na/bo/gi/ga yeg/gye/ue/\\
      ga/sir/sday/ey/nvn/\\
      jug/e/do/a/ni nun/mur/hvr/ni/u/ri/da}
\end{enumerate}
\end{document}
